\section{Research Questions and Hypotheses}
We will investigate the likely start of sentence production by using a model that freely makes decisions. We do this by pre-biasing the utility of the versions of rules that are at various points in the sentence. We compare several initial utility distributions. 

\begin{itemize}
\item The \textit{Incremental} model has an initial utility distribution that favors syntactic rules at the beginning of the sentence, decreasing for rules later in the sentence.
\item The \textit{Centered} model has an initial distribution favoring rules in the center of the sentence, decreasing as they get closer to the end and the beginning.
\item The \textit{Reverse} model has an initial distribution favoring rules at the end of the sentence, decreasing as they get closer to the beginning.
\item The \textit{Control} model has no initial utility distribution, with every rule starting at zero utility.
\end{itemize}

As a reminder, all of these utilities would change as the model progresses and makes sentences. However, the \textit{Control} model has certain disadvantages. As there is an initial sentence the model receives, having any imposed structure would make the model more likely to produce an utterance closer to that original sentence, which is presumably slightly more plausible than any given grammatical utterance. Nonetheless, we hypothesize due to a decreased cognitive load and an increased flexibility in the usage of syntactic rules that the \textit{Incremental} model will perform the best.