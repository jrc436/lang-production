\section{Introduction}
 Studying language production in an effective way thus far has been hard. Psycholinguistics generally relies on carefully controlled experiments; however, studying production in this way is difficult because participants are presented with a large degree of choice. For example, if you ask a subject to describe a picture without controlling what he or she will say, there are a (theoretically) infinite number of utterances he or she could produce. Due to the difficulty in collecting large amounts of data in such a setting, comparing a massive number of experimental conditions is not really feasible. 
 
 For instance, consider an experimenter who wishes to contrast the choice of a Double Object construction (The boy threw his dog the ball) with a Prepositional Object construction (The boy threw the ball to his dog). If the participant is simply describing a picture, there is no guarantee he or she will even reference the dog! The primary strategy for countering this problem has been to provide the participant with the syntactic structure of the sentence \citep{sums-incr} or with the first few words of the utterance \citep{prod-exp}. In the latter of these studies, the researchers still had to include an "Other" category for sentences that did not fit in any of the desired conditions. 
 
 While these controls have been useful, they could have the side effect of muddling certain parts of the process: such as planning. In order to produce a sentence, it must first be planned. Investigating the planning process is difficult: many paradigms frequently used to investigate sentence comprehension, such as eye tracking or the visual world, seem much more difficult to apply. Despite the difficulty in designing experiments to measure planning, it is clearly of interest to researchers, as can be evidenced by ongoing debates. One such debate is to what degree is the process \emph{incremental}. 
 
 For instance, with sentences that contain center-embedded clauses, is the end planned before or after the embedded clause? Consider the sentence "The dog that was chasing the cat that seemed to have rabies fell over." An incremental derivation would rely on \emph{surface order}: the actual order the words appear in the sentence. In other words, the speaker first plans "the dog", then "that was chasing the cat", then "that seemed to have rabies" and lastly "fell over." A less incremental derivation might be to plan "fell over" immediately after "the dog", and then to add the clauses later.
 
These questions are controversial, prompting ideas without clear cut answers. Due to the seeming difficulty in solving this debates such as these with more experimentation, we suggest turning to computational cognitive modeling.
 
%