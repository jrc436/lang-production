\section{Previous Work}
Language processing, in terms of both comprehension and production, have been explored broadly by the cognitive modeling community.

In terms of comprehension, cognitive models have been created both to explain several phenomenon and more generally. \citet{decision} explained the lexical decision task as a by-product of chunk activation. \citet{anaphoric} provides evidence that memory retrieval is likewise sufficient to explain whether nouns are treated as anaphoric: whether they refer to an antecedent or are a new reference. Both \citet{comp-model} and \citet{big-comprehension} make strides toward more general models of language comprehension.

The models to explain language production are thus far, more narrow. \citet{references} created a model that produced references for the iMAP task. \citet{model} was a cognitive model of syntactic priming; it demonstrated that priming can be explained by activation. It made linguistic choices, but only well-defined choices, such as whether to choose double object or prepositional object constructions. 

While broader coverage models of language generation do exist \citep{chart}, they make no claims of cognitive plausibility. Thus, we see this paper as a step towards a broad, realistic model of language production. 