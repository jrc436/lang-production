\subsection{ACT-R}
ACT-R has a few basic units of organization. The primary units for declarative memory are \textit{chunks}. A chunk is a fairly simple concept that basically refers to one thing that can be retrieved or held in working memory. Chunks can be arbitrarily simple or complicated by way of \textit{slots}. A slot is a simple data type that corresponds to another chunk. The most primitive chunks thus have a name, but no slots. 

Buffers are a unit of cognition that can hold exactly one chunk. While ACT-R can, in theory, have an arbitrary number of buffers (including those such as the visual buffer or the auditory buffer), we make use of only the simplest buffers: the retrieval buffer and the goal buffer. The retrieval buffer can be thought of as the state of memory retrieval. It can be empty or it can be retrieving something or it can contain something it just retrieved. The goal buffer, on the other hand, can be thought of as working memory. It also can only contain one chunk, but it contains the state of the problem that is being solved. 

Production rules are the allowable manipulations of the buffers, and the conditions upon which those manipulations should be performed. For instance, a model could have a production rule that says to retrieve something when the goal buffer is at some state. It might, for instance, call for looking up the syntactic type of the word that's currently in a given slot in the goal buffer. These production rules, in combination with chunks, form a cognitive model of some task when by meeting these conditions iteratively, they can successfully transform the initial state of the problem into its goal state. 

ACT-R additionally has a couple of other mechanisms. One is a form of simple utility learning, where after a rule fires, it is rewarded, penalized, or neither. We will use this type of learning with the tracing model. Another is activation, which affects which matching chunk is retrieved. This activation can be used to explain many linguistic phenomenon, such as priming \citep{priming} \citep{model}.  

In the following, we'll present the full set of chunks and production rules that make up our model. 