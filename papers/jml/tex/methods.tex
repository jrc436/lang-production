\section{Methods}
While we use the bigram method previously described to evaluate the utterances described, we also need something to compare to. To do this, we put randomly sampled Switchboard sentences through the same evaluation scheme. We additionally compare variants of the model to this condition, to see which one best fits the data. As described earlier, our primary research question is to determine the degree that language production is incremental. We believe the effectiveness of each model in fitting the human data can provide evidence toward answering that question.

\begin{itemize}
\item The \textit{Incremental} model has an initial utility distribution that favors syntactic rules at the beginning of the sentence, decreasing for rules later in the sentence.
\item The \textit{Centered} model has an initial distribution favoring rules in the center of the sentence, decreasing as they get closer to the end and the beginning.
\item The \textit{Reverse} model has an initial distribution favoring rules at the end of the sentence, decreasing as they get closer to the beginning.
\item The \textit{Default} model has no initial utility distribution, with every rule starting at zero utility.
\end{itemize}

As a reminder, all of these utilities would change as the model progresses and makes sentences. However, the \textit{Default} model has certain disadvantages. As there is an initial sentence the model receives, having any imposed structure would make the model more likely to produce an utterance closer to that original sentence, which is presumably slightly more plausible than any given grammatical utterance. Nonetheless, we hypothesize due to a decreased cognitive load and an increased flexibility in the usage of syntactic rules that the \textit{Incremental} model will perform the best.

Each of the four models and the control condition were fed ten groups of one hundred sentences. Each of them is plotted and compared by their standard deviation, mean, and median.