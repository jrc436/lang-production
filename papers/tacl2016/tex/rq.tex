\section{Research Questions}
To validate our methodology, we chose a few research questions that have been discussed, but not definitively answered in the literature.

\begin{itemize}
	\item{ \textbf{Are sentences produced incrementally easier or more difficult to produce?} \citet{sums-incr} suggested that while under heavier stress, participants relied on more incremental constructions. However, in their study, they told participants whether to respond incrementally or non-incrementally, by providing a template. Further, the individual steps of language production can't be untangled. Moreover, by the simple nature of syntactic analysis, some sentences will be more difficult to produce incrementally, while some sentences will be easier, providing an additional lurking variable. Thus, by evaluating this sentence on the corpus, without biasing the model to be more or less incremental, we can provide additional evidence toward answering this question. }
	\item{ \textbf{Does increasing the amount of working memory available for the task lead to a corresponding increase in task success?} While some have theorized about the nature of working memory in language production \citep{vwm}, and others have asked similar questions for certain processes of comprehension, such as reading \citep{wmread}, to our knowledge, the effect of working memory capacity has not been similarly investigated in language production. However, even if it had, it would likely be susceptible to many of the same weaknesses as other experiments about language production: the stages are difficult to isolate and it is difficult to control all variables. Moreover, an additional perspective is useful regardless of one's confidence in various experimental paradigms.}
\end{itemize}