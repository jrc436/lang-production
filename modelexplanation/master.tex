%
% File acl2015.tex
%
% Contact: car@ir.hit.edu.cn, gdzhou@suda.edu.cn
%%
%% Based on the style files for ACL-2014, which were, in turn,
%% Based on the style files for ACL-2013, which were, in turn,
%% Based on the style files for ACL-2012, which were, in turn,
%% based on the style files for ACL-2011, which were, in turn, 
%% based on the style files for ACL-2010, which were, in turn, 
%% based on the style files for ACL-IJCNLP-2009, which were, in turn,
%% based on the style files for EACL-2009 and IJCNLP-2008...

%% Based on the style files for EACL 2006 by 
%%e.agirre@ehu.es or Sergi.Balari@uab.es
%% and that of ACL 08 by Joakim Nivre and Noah Smith

\documentclass[11pt]{article}
\usepackage{acl2015}
\usepackage{times}
\usepackage{url}
\usepackage{latexsym}
\usepackage{amsmath, amssymb}



\usepackage[american]{babel}
%\DeclareLanguageMapping{american}{american-apa}

%% genitive citations
%\newcommand{\citegen}[1]{\citeauthor{#1}'s \citeyear{#1}}
 \newcommand{\citep}[1]{\cite{#1}}
% \newcommand{\citet}[1]{\citeA{#1}}
% \newcommand{\citealt}[1]{\citeNP{#1}}



%\setlength\titlebox{5cm}

% You can expand the titlebox if you need extra space
% to show all the authors. Please do not make the titlebox
% smaller than 5cm (the original size); we will check this
% in the camera-ready version and ask you to change it back.


\title{An ACT-R Model of Language Production}

\author{Jeremy Cole \\
College of IST \\
Pennsylvania State University \\
University Park, PA, 16802, USA \\
  {\tt jrcole@psu.edu} \\ \And
David Reitter \\
College of IST \\
Pennsylvania State University \\
University Park, PA, 16802, USA \\
  {\tt reitter@psu.edu} \\
}

\date{February 25, 2016}


\begin{document}
\maketitle
\begin{abstract}
We present a model of language production in a way that can be considered cognitively plausible but adaptable enough to consider among a range of possibilities. While this model will produce naturalistic output, it does not consider realistic output in a vacuum to be useful. Thus, it will produce the best output it can while maximizing cognitive plausibility. Nonetheless, between two models that are cognitively plausible, we will consider the one producing more human-like output to be more human-like. Our goal is thus to use cognitive modeling to answer several questions about how language production actually works.
\end{abstract}

\section{Introduction}
 Studying language production in an effective way thus far has been hard. Psycholinguistics generally relies on carefully controlled experiments; however, studying production in this way is difficult because participants are presented with a large degree of choice. For example, if you ask a subject to describe a picture without controlling what he or she will say, there are a (theoretically) infinite number of utterances he or she could produce. Due to the difficulty in collecting large amounts of data in such a setting, comparing a massive number of experimental conditions is not really feasible. \\
 \indent For instance, consider an experimenter who wishes to contrast the choice of a Double Object construction (The boy threw his dog the ball) with a Prepositional Object construction (The boy threw the ball to his dog). If the participant is simply describing a picture, there is no guarantee he or she will even reference the dog! The primary strategy for countering this problem has been to provide the participant with the syntactic structure of the sentence \citep{incremental} or with the first few words of the utterance \citep{chinese}. In the latter of these studies, the researchers still had to include an "Other" category for sentences that did not fit in any of the desired conditions. \\
 \indent While these controls have been useful, they could have the side effect of muddling certain parts of the process: such as planning. In order to produce a sentence, it must first be planned. Investigating the planning process is difficult: many paradigms frequently used to investigate sentence comprehension, such as eye tracking or the visual world, seem much more difficult to apply. Despite the difficulty in designing experiments to measure planning, it is clearly of interest to researchers, as can be evidenced by ongoing debates. One such debate is to what degree is the process \emph{incremental}. \\
 \indent For instance, with sentences that contain center-embedded clauses, is the end planned before or after the embedded clause? Consider the sentence "The dog that was chasing the cat that seemed to have rabies fell over." An incremental derivation would rely on \emph{surface order}: the actual order the words appear in the sentence. In other words, the speaker first plans "the dog", then "that was chasing the cat", then "that seemed to have rabies" and lastly "fell over." A less incremental derivation might be to plan "fell over" immediately after "the dog", and then to add the clauses later. \\
 \indent Some researchers seem to take it as a given that language processes are radically incremental (especially with regards to comprehension) \citep{tag1}\citep{radical}. However, the evidence (especially for production) is not so clear \citep{incremental}. Due to the seeming difficulty in solving this debates such as these with more experimentation, we suggest turning to computational cognitive modeling.\\
 \indent Computational cognitive models of language production are not novel in of themselves \citep{model}. However, to some extent, they suffer from the same problem as experimentation. The breadth of linguistic diversity falls short of the sentences that any experimenters can plan for. To solve these problems, we need computational cognitive models with wide coverage. We believe our model is a step in this direction.
 
\section{Goals}
Our goal is to achieve this using the tight constraints of the ACT-R system \citep{actr}. We further attempt to stick within the constraints of plausibility by not relying on strategies such as resampling from previously heard input, due to its inability to produce novel productions. Instead, we define a range of operations that are defined in the Combinatory Categorial Grammar (CCG) \citep{ccg}, a minimally context sensitive grammar that is believed to be capable of all of the syntactic operations of human language \citep{convergence}. However, while our current system uses the operations of CCG, our theoretical basis of combinatory operations does not explicitly rely on CCG. Indeed, our system could in the future compare two combinatory systems for plausibility. Our longterm goals take nothing for granted: any operation, method, or structure that is cognitively plausible, fits the data, and produces realistic output is a possibility. 

In the short term, however, we make several assumptions about the operations and structure of language production. The model has two basic modes to accomplish this. One can be thought of as \textit{naturalistic} mode. In naturalistic mode, the model attempts to combine whatever words it wants, given its current settings, rules, chunks, and goals. In this mode, it is natural for the model to make some mistakes, but it can still be compared with itself for \textit{plausibility}. The second mode is \textit{tracing} mode. In tracing mode, the model will be corrected if it attempts to produce something incorrect. Then, the number of mistakes made are tallied over time. While the number of mistakes will still be high, they should decrease over time. In this way, we are comparing two models for \textit{learnability}. 

\section{Evaluation}
The model receives as input a set of words that made up a sentence in the Switchboard corpus. The model will then, using its knowledge about these words and possible combinatory operations, attempt to produce a sentence or sentence fragment. Each mode will have a different evaluation metric, and the questions we hope to ask are also slightly different. In tracing mode, the goal of the model is to reduce the number of mistakes by more than the control model. We expect model that learns more quickly to be a better model of language production than one which has more trouble learning. Thus, we will compare error differences after several benchmarks between the two models to determine which is more learnable. In naturalistic mode, the goal is to be more plausible. Thus, we will compute the average ngram scores using the SRILM toolkit \citep{srilm} for the utterances of each model. As the models are generating from the same bag of words, there should be no disadvantage to choosing uncommon but naturalistic phrases, especially over a large dataset. However, there is a possibility longer utterances are naturally less probable, so some adjustment may be necessary. Further, computing ngram scores for ngrams larger than five words is fairly unrealistic with current technology. It is possible that some related metric, such as the average of all 3-gram scores, could solve both of these problems. 

\section{ACT-R}
ACT-R has a few basic units of organization. The primary units for declarative memory are \textit{chunks}. A chunk is a fairly simple concept that basically refers to one thing that can be retrieved or held in working memory. Chunks can be arbitrarily simple or complicated by way of \textit{slots}. A slot is a simple data type that corresponds to another chunk. The most primitive chunks thus have a name, but no slots. 

Buffers are a unit of cognition that can hold exactly one chunk. While ACT-R can, in theory, have an arbitrary number of buffers (including those such as the visual buffer or the auditory buffer), we make use of only the simplest buffers: the retrieval buffer and the goal buffer. The retrieval buffer can be thought of as the state of memory retrieval. It can be empty or it can be retrieving something or it can contain something it just retrieved. The goal buffer, on the other hand, can be thought of as working memory. It also can only contain one chunk, but it contains the state of the problem that is being solved. 

Production rules are the allowable manipulations of the buffers, and the conditions upon which those manipulations should be performed. For instance, a model could have a production rule that says to retrieve something when the goal buffer is at some state. It might, for instance, call for looking up the syntactic type of the word that's currently in a given slot in the goal buffer. These production rules, in combination with chunks, form a cognitive model of some task when by meeting these conditions iteratively, they can successfully transform the initial state of the problem into its goal state. 

ACT-R additionally has a couple of other mechanisms. One is a form of simple utility learning, where after a rule fires, it is rewarded, penalized, or neither. We will use this type of learning with the tracing model. Another is activation, which affects which matching chunk is retrieved. This activation can be used to explain many linguistic phenomenon, such as priming \citep{priming} \citep{model}.  

In the following, we'll present the full set of chunks and production rules that make up our model. 
\section{Chunks and Buffers in the Model}
Our model relies on a fairly simple set of chunks. 

\subsection{Word}
A word is the simplest type. It just has a name, which is the text of the word. It has no slots. The word chunks are built by creating a chunk of each word in the Switchboard corpus.

\subsection{Type}
At the heart of our model are chunks representing CCG syntactic types. There are two basic types of CCG types, though they appear to ACT-R as the same. One is what we refer to as a \textit{simple} type, where the type is a primitive type, such as a noun phrase. There are approximately ten basic CCG types found in the Switchboard data set. The second is a \textit{compound} type, where the type is composed of a left type, a right type, and a \textit{combinatory operator}. The left type and right type of each compound type can be either simple or additional compound types. Type chunks thus have three slots, the FullType, the LeftType, the RightType, and the Combinatory Operator (the last three are null if it is a simple type). Lastly, it has a flag to mark if that type is able to join (via conjunctions) with others of its type. The type chunks are built by creating a type for each type that appears in the Switchboard corpus.

\subsection{Combinatory Operator}
A combinatory operator is a very simple operator that can be thought of as a dependency within a type. As the end-goal of sentence production is to produce a sentence, a complete sentence is always of type \textit{S}, which is one of the CCG simple types. Then, verbs, adjectives, and so on are assigned compound types in such a way that they could ultimately be part of a complete sentence. For instance, an adjective might be NP/N, suggesting that it needs a noun to its right in order to be a noun phrase. If it was NP\textbackslash N, it would instead suggest it needs a noun to its left. Thus, the two combinatory operators are slash and backslash. Combinatory Operator chunks have no slots. These two combinatory operators are predefined.

\subsection{Lexsyn}
A Lexsyn is a Chunk that associates words with their type information. It, simply put, has a slot for the word, and then has the five slots for each type the word can take, where these slots are the exact same as described in the Type chunk. These are built by reading Switchboard, and associating every type (including type-raised types) that any given word ever takes with the word, and then creating the Lexsyn accordingly.  

\subsection{Sentence}
The Sentence is essentially the goal chunk. In this way, it basically contains predefined slots that are the input, and then predefined slots to store the output. Its first several slots is a set of words that are unordered. These are predefined, and consist of the words in a sentence that appears in Switchboard. The Sentence chunk also has several other slots, which correspond basically to lexsyns. Due to the methodology for accessing slots of slots, these are all stored in a flat manner, similarly to how the lexsyn stores types. Thus, the sentence chunk's total number of slots is the maximum number of words in the target sentences multiplied by the maximum number of types any given word has in the target sentences. This makes the chunk quite large. 

The final, correct output for the model is a single Lexsyn with the type S. In general, the output can be thought of as a Lexsyn that contains some number of words and some final syntactic type, where it can no longer combine with any of the other lexsyns. Details on how the Rules utilize the working space in the Sentence chunk will be covered in the next section. 

\section{Production Rules}
\subsection{Retrieving Syntactic Information}
There are two relevant production rules to begin processing: GrabWord and AddLexSyn. GrabWord is the simple process of retrieving syntactic information about a word in the input. It can fire at any time the retrieval buffer is not busy. This takes a word from the input and makes a retrieval request for a Lexsyn with that word. AddLexSyn then updates the goal state and does cleanup, by removing the word from input and adding the lexsyn to the goal buffer. This presently allows for an infinite working memory span, though the model in principle could work the same by having a fixed working memory span. As of now, however, this could possibly lead to retrieving a set of lexsyns where none could be combined. Both GrabWord and Lexsyn have as many variants as there are maximum number of words in a sentence. 

\subsection{Syntactic Rules and Resolutions}
Each Syntactic rule is defined many times, requiring dependencies in the Sentence chunk to be met. Mostly, this is by checking if any Lexsyns in the sentence chunk are matches for each other's dependencies. If it is a match, a retrieval request is put out for the matching type. This sets up another type of production rule to follow where the rule is resolved. This updates the type that's stored in chunk, removes the chunk that's been absorbed (in Composition, for no particular reason, the left always absorbs the right), and removes all other possible types of the new chunk, if they still exist. In other words, on the first combination, each Lexsyn contains multiple types, but after it's been combined and forms a type that's not specific to that word, the other types are removed as they are non-equivalent. The syntactic rules in CCG are described below. 

\subsubsection{Forward Application}
Forward Application is a simple combinatory rule where a single dependency of a compound type is resolved to the right (in other words, the type has the Slash combinatory operator). The type resolving the dependency can be a simple type or a compound type. 
\subsubsection{Backward Application}
Backward Application is a simple combinatory rule where a single dependency of a compound type is resolved to the left (in other words, the type has the Backslash operator). The type resolving the dependency can be a simple type or a compound type.
\subsubsection{Forward Composition}
Forward Composition takes two compound types with the Slash operator. If one type has a dependency to the right, and the other type's left side resolves that dependency, then that type gains the other type's right side as a new dependency to the right but resolves its current one and adds that word to the right. 
\subsubsection{Backward Composition}
Backward Composition takes two compound types with the Backslash operator. If one type has a dependency to the left, and the other type's right side resolves that dependency, then that type gains the other type's left side as a new dependency to the left but resolves its current one and adds that word to the left.  
\subsubsection{BeginConjunction}
This corresponds to adding a conjunction to a type that could benefit from it. For instance, this would be the process of transforming "walking" to "and walking" or "walking and". 
\subsubsection{FinishConjunction}
This corresponds to adding an equivalent type to the phrase that contains the conjunction. In other words, it's the processing of transforming "and walking" to "talking and walking" or "walking and" to "walking and talking".

\section{Research Questions}
To validate our methodology, we chose a few research questions that have been discussed, but not definitively answered in the literature.

\begin{itemize}
	\item{ \textbf{Are sentences produced incrementally easier or more difficult to produce?} \citet{sums-incr} suggested that while under heavier stress, participants relied on more incremental constructions. However, in their study, they told participants whether to respond incrementally or non-incrementally, by providing a template. Further, the individual steps of language production can't be untangled. Moreover, by the simple nature of syntactic analysis, some sentences will be more difficult to produce incrementally, while some sentences will be easier, providing an additional lurking variable. Thus, by evaluating this sentence on the corpus, without biasing the model to be more or less incremental, we can provide additional evidence toward answering this question. }
	\item{ \textbf{Does increasing the amount of working memory available for the task lead to a corresponding increase in task success?} While some have theorized about the nature of working memory in language production \citep{vwm}, and others have asked similar questions for certain processes of comprehension, such as reading \citep{wmread}, to our knowledge, the effect of working memory capacity has not been similarly investigated in language production. However, even if it had, it would likely be susceptible to many of the same weaknesses as other experiments about language production: the stages are difficult to isolate and it is difficult to control all variables. Moreover, an additional perspective is useful regardless of one's confidence in various experimental paradigms.}
\end{itemize}
\section{Conclusion}
In this paper, we have presented an ACT-R model of language production capable of producing novel sentences and sentence fragments. It takes as input a set of words, having foreknowledge of both a larger set of words and the syntactic types those words can be. Then, it attempts to combine these words using the rules defined in CCG. The model has two modes, one where errors are measured over time (learnability) as the model attempts to produce a sentence that exists in the Switchboard corpus, and another where the plausibility of whatever the model did produce is measured. This allows us to ask a variety of interesting scientific questions of the model that previous studies have been unable to answer.

\bibliographystyle{acl}
\bibliography{newbib}

\end{document}
