\section{Conclusion}
In this paper, we presented a methodology for pursuing psycholinguistic research. Borrowing from linguistic theory and computational modeling, we discussed how cognitive models, such as those implemented in ACT-R, can move beyond hand-coded models and pre-determined datasets. Further, we discuss how our model can disentangle somewhat complicated phenomena due to the ease with which several important variables, such as the capacity of working memory, can be manipulated. In particular, we find that sentences that require less working memory are easier to realize, but that having additional working memory available improves overall performance. We further find that performance increases as sentences are more right-branching, which we see as in line with previous research.