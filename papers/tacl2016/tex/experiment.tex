\section{Experiment 2}
To determine whether incrementality and working memory were correlated with performance by chance, we performed another experiment where we varied the working memory available to the model. The model in experiment 1 had five slots available for it to store lexsyns, so we compared it to a model with only three slots available to gain further evidence that increased working memory does not improve performance. All metrics were computed using the same method as in Experiment 1, though Experiment 2 was also compared to Experiment 1 for its relative edit distance. The correlations with edit distance can be found in Table~\ref{exp2}. The comparison with Experiment 1 can be found in Table~\ref{compare}.

\subsection{Results}
We found all metrics highly significant, as in Experiment 1. As Experiment 2 realized the exact same sentences as Experiment 1, we are able to use paired t-tests to see the differences between them on various variables. 

\begin{table}
	\centering
	\begin{tabular}{l|cccc}
		& WBF & UBF & WMU & AWMU \\ \hline
		p-value & $<0.0001$ & $<0.0001$ & $<0.0001$ & $<0.0001$ \\
		effect & $-0.162$ & $-0.228 $ & $0.120$ & $0.041$ \\
	\end{tabular}
	\label{exp2}
	\caption{Correlations as individual linear models with edit distance}
\end{table}

\begin{table}
	\centering
	\begin{tabular}{l|ccccc}
		           & WBF & UBF & WMU & AWMU & dist \\ \hline
		Exp1-Mean & $1.050$ & $0.132$ & $3.156$ & $1.721$ & $0.743$ \\
		Exp2-Mean & $1.023$ & $0.105$ & $2.703$ & $1.575$ & $0.748$ \\
		p-value   & $<0.0001$  & $<0.0001$ & $<0.0001$ & $<0.0001$ & $<0.0001$ \\
	\end{tabular}
	\label{compare}
	\caption{Paired t-tests between Experiment 1 and Experiment 2}
\end{table}

As can be seen, the model with more working memory performs slightly better, even though it uses more working memory on average by both metrics. However, it is also more right-branching than the other model by both metrics.

\subsection{Discussion}
The most important takeaway is that having working memory available when needed clearly improves performance, even though in general, using more working memory worsens performance. Besides that, varying the capacity of working memory does not change the general strategy of grammatical encoding, which prefers to use less working memory and more right-branching constructions. Still, the model with less working memory was less right-branching. This could perhaps be because without additional working memory available, it sometimes had to settle for an inferior strategy, perhaps explaining its performance decline.